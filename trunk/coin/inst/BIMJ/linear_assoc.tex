
\documentclass[bimj,fleqn]{w-art}
\usepackage{times}
\usepackage{w-thm}
\usepackage[authoryear]{natbib}
\setlength{\bibsep}{2pt}
\setlength{\bibhang}{2em}


\usepackage{Sweave}

\usepackage{verbatim}
\usepackage{graphicx}
\usepackage{amsfonts}
\usepackage{amstext}
\usepackage{amsmath}
%%\usepackage{amsthm}
%%\usepackage[round]{natbib}
%%\usepackage{bibentry}
%%\usepackage{hyperref}
\usepackage{thumbpdf}
\usepackage{rotating}
%%\usepackage{floatflt}

\newcommand{\R}{\mathbb{R} }
\newcommand{\Prob}{\mathbb{P} }
\newcommand{\N}{\mathbb{N} }
\newcommand{\C}{\mathbb{C} }
\newcommand{\V}{\mathbb{V}} %% cal{\mbox{\textnormal{Var}}} }
\newcommand{\E}{\mathbb{E}} %%mathcal{\mbox{\textnormal{E}}} }
\newcommand{\Var}{\mathbb{V}} %%mathcal{\mbox{\textnormal{Var}}} }
\newcommand{\argmin}{\operatorname{argmin}\displaylimits}
\newcommand{\argmax}{\operatorname{argmax}\displaylimits}
\newcommand{\LS}{\mathcal{L}_n}
\newcommand{\TS}{\mathcal{T}_n}
\newcommand{\LSc}{\mathcal{L}_{\text{comb},n}}
\newcommand{\LSbc}{\mathcal{L}^*_{\text{comb},n}}
\newcommand{\F}{\mathcal{F}}
\newcommand{\A}{\mathcal{A}}
\newcommand{\yn}{y_{\text{new}}}
\newcommand{\z}{\mathbf{z}}
\newcommand{\X}{\mathbf{X}}
\newcommand{\Y}{\mathbf{Y}}
\newcommand{\sX}{\mathcal{X}}
\newcommand{\sY}{\mathcal{Y}}
\newcommand{\T}{\mathbf{T}}
\newcommand{\x}{\mathbf{x}}
\renewcommand{\a}{\mathbf{a}}
\newcommand{\xn}{\mathbf{x}_{\text{new}}}
\newcommand{\y}{\mathbf{y}}
\newcommand{\w}{\mathbf{w}}
\newcommand{\ws}{\mathbf{w}_\cdot}
\renewcommand{\t}{\mathbf{t}}
\newcommand{\M}{\mathbf{M}}
\renewcommand{\vec}{\text{vec}}
\newcommand{\B}{\mathbf{B}}
\newcommand{\K}{\mathbf{K}}
\newcommand{\W}{\mathbf{W}}
\newcommand{\D}{\mathbf{D}}
\newcommand{\I}{\mathbf{I}}
\newcommand{\bS}{\mathbf{S}}
\newcommand{\cellx}{\pi_n[\x]}
\newcommand{\partn}{\pi_n(\mathcal{L}_n)}
\newcommand{\err}{\text{Err}}
\newcommand{\ea}{\widehat{\text{Err}}^{(a)}}
\newcommand{\ecv}{\widehat{\text{Err}}^{(cv1)}}
\newcommand{\ecvten}{\widehat{\text{Err}}^{(cv10)}}
\newcommand{\eone}{\widehat{\text{Err}}^{(1)}}
\newcommand{\eplus}{\widehat{\text{Err}}^{(.632+)}}
\newcommand{\eoob}{\widehat{\text{Err}}^{(oob)}}

\hyphenation{Qua-dra-tic}

\newcommand{\Rpackage}[1]{{\normalfont\fontseries{b}\selectfont #1}}
\newcommand{\Robject}[1]{\texttt{#1}}
\newcommand{\Rclass}[1]{\textit{#1}}
\newcommand{\Rcmd}[1]{\texttt{#1}}
\newcommand{\Roperator}[1]{\texttt{#1}}
\newcommand{\Rarg}[1]{\texttt{#1}}
\newcommand{\Rlevel}[1]{\texttt{#1}}

\newcommand{\RR}{\textsf{R}}
\renewcommand{\S}{\textsf{S}}



%% end of declarations %%%%%%%%%%%%%%%%%%%%%%%%%%%%%%%%%%%%%%%%%%%%%%%


\begin{document}

\newcommand{\url}[1]{\texttt{#1}}

\DOIsuffix{bimj.DOIsuffix}
\Volume{50}
\Issue{3}
\Year{2008}

\pagespan{1}{}

\Receiveddate{xx February 2008}
\Reviseddate{xx March 2008}
\Accepteddate{xx April 2008}
\Dateposted{xx April 2008}
%%
\keywords{genetic association, case-control study, robust trend test,
maximum test, conditional inference.
}

\title[Order-restricted Scores Test]{Order-restricted Scores Test for the Evaluation of \\
       Population-based Case-control Studies when the \\
       Genetic Model is Unknown}

\author[Ludwig A. Hothorn]{Ludwig A. Hothorn \inst{1}}
\address[\inst{1}]{Institut f\"ur Biostatistik, Leibniz Universit\"at Hannover \\
Herrenh\"auser Stra{\ss}e 2, D--30419 Hannover, Germany}

\author[Torsten Hothorn]{Torsten Hothorn\footnote{Corresponding
     author: e-mail: {\sf Torsten.Hothorn@stat.uni-muenchen.de}, Phone: +49\,89\,21806407,
     Fax: +49\,89\,21805040.} \inst{2}} 
\address[\inst{2}]{
Institut f{\"u}r Statistik \\ Ludwig-Maximilians-Universit{\"a}t M{\"u}nchen \\
Ludwigstra{\ss }e 33, D--80539 M{\"u}nchen, Germany}


\begin{abstract}
The Cochran-Armitage linear trend test for proportions is often 
used for genotype-based analysis of candidate gene association. Depending on 
the underlying genetic mode of inheritance, the use of model-specific scores maximises 
the power. Commonly, the underlying genetic model, i.e.,
additive, dominant or recessive mode of inheritance, is a priori unknown.
Association studies are commonly analysed using permutation tests, where
both inference and identification of the underlying mode of inheritance are important.
Especially interesting are tests for case-control studies, 
defined by a maximum over a series of standardised Cochran-Armitage tests, because 
such a procedure has power under all three genetic models.

We reformulate the test problem and propose a conditional maximum test 
of scores-specific linear-by-linear association tests \citep{bookagresti2002}. For maximum-type, sum
and quadratic test statistics the asymptotic expectation and covariance 
can be derived in a closed form and the limiting distribution
is known. Both the limiting distribution and approximations of the
exact conditional distribution can easily be computed using standard
software packages. In addition to these technical advances,
we extend the area of application to stratified designs,
studies involving more than two groups and the simultaneous analysis
of multiple loci by means of multiplicity-adjusted $p$-values for the
underlying multiple Cochran-Armitage trend tests. The new test is applied
to reanalyse a study investigating genetic components of 
different subtypes of psoriasis.

A new and flexible inference tool for association
studies is available both theoretically as well as practically since 
already available software packages can be easily used to implement
the suggested test procedures.
\end{abstract}

\maketitle

\section{Objectives}

In population-based case-control studies the association between a 
candidate allele and a disease can be evaluated by the Cochran-Armitage 
(CA) trend test \citep{Armitage:1955}, regardless of
whether or not Hardy-Weinberg equilibrium holds 
\citep{Sasieni:1997}. 
The CA test is based on a set of scores assigned to the alleles. 
For genotypes $aa$, $Aa$, or $AA$, with $A$ denoting a high risk 
candidate allele and $a$ any of the other alleles, 
three-dimensional score vectors optimising the power of the CA test
against dominant, additive, and recessive alternatives can be defined. 
If the underlying mode of inheritance is known, the choice of
an appropriate score vector for the trend test is obvious. 
However, in situations where the underlying genetic model is unknown
choosing the \textit{wrong} score vector leads to a substantial loss of power
as shown by \citet{Freidlin:2002}.
Therefore, inference procedures
with good power under \textit{all} three genetic models 
are of special interest.
An appealing approach is to construct a test based on all three possible
trend tests, for example utilising the maximum of the standardised 
test statistics of the CA tests which are optimal under the dominant, 
additive, and recessive model.
An unconditional version of this test was proposed and investigated by \citet{Freidlin:2002}. 
The distribution of this maximum test, called MAX test hereafter,
under the null hypothesis of equal genotype distribution
in cases and controls is approximated by simulation procedures
by \citet{Freidlin:2002}
since the unconditional asymptotic distribution is hard to derive.

In this paper, we show that the conditional MAX test is a simple special
case of a general class of linear statistics whose conditional
distribution is easy to compute. Thus, without additional
theoretical effort, the MAX test enjoys all the nice properties
of this class of conditional tests. Therefore, we embed the MAX test as suggested by 
\citet{Freidlin:2002} into the flexible framework for 
conditional independence tests introduced by 
\citet{StrasserWeber1999}. The merits of doing so are
i) the conditional reference distribution for the MAX test can be easily 
approximated either by evaluating a three-dimensional normal distribution
or by conditional Monte-Carlo experiments,
ii) tests for stratified designs, designs with more than two
groups and two or more loci can be defined in a rather 
straightforward way, iii) the most likely underlying
mode of inheritance can be estimated by
multiplicity-adjusted $p$-values for the
three CA statistics under consideration, iv) the simultaneous analysis
of multiple loci, and v) the analysis of 
genetic association studies using the MAX test and its newly
introduced extension can be performed by already
available software implementations of the 
\citet{StrasserWeber1999} framework.

\section{Maximum Test}

For case-control studies investigating the association between a binary phenotype 
and the measured alleles of a candidate gene the data are typically 
presented as empirical genotype distribution in each group. 
For a simple bi-allelic marker the data can be 
presented in a $3 \times 2$ contingency table, where $A$ is the 
high risk candidate allele and $a$ is any of the other alleles 
(see Table~\ref{gdistr}). 

\begin{table}
\begin{center}
\caption{Genotype distributions for cases and controls. \label{gdistr}}
\vspace*{0.5cm}
  \begin{tabular}{l l l l}
       & Cases & Controls & Total \\
 \hline
 $aa$       &       $r_{aa}$ & $s_{aa}$ & $n_{aa}$\\
 $aA$       &       $r_{aA}$ & $s_{aA}$ & $n_{aA}$ \\
 $AA$       &       $r_{AA}$ & $s_{AA}$ & $n_{AA}$ \\
 Total&         $R$     & $S$       & $N$  \\
  \hline
 \end{tabular}
\end{center}
 \end{table}


We are interested in a comparison of the genotype distributions, 
i.e., the penetrances $f_j = P(\text{case} | j)$ for $j \in \{aa, aA, AA\}$:
\begin{eqnarray*}
H_0: f_{aa} = f_{aA} = f_{AA} \text{ vs. } H_1: f_{aa} \le f_{aA} \le f_{AA}
\end{eqnarray*}
where at least one of the inequalities in the alternative is strict.

The CA test statistic with scores 
$\mathbb{\xi} = (\xi_{aa}, \xi_{aA}, \xi_{AA})$ 
is essentially (modulo standardisation)
\begin{eqnarray} \label{CAstat}
\text{CA}(\mathbb{\xi}) = \sum_{j \in \{aa, aA, AA\}} \xi_j r_j.
\end{eqnarray}
If the mode of inheritance is best described by the dominant model,
the scores $\mathbb{\xi}_\text{dom}=(0,1,1)$ 
($f_{aa} < f_{aA} = f_{AA}$) will lead to a trend test with maximal power.
Under the recessive model the score vector 
$\mathbb{\xi}_\text{rec}=(0,0,1)$ ($f_{aa} =  f_{aA} < f_{AA}$)
is power optimal whereas a linear trend represented
by scores $\mathbb{\xi}_\text{add}=(0,1,2)$ should be chosen
when the mode if inheritance is additive, i.e., $f_{aa} < f_{aA} < f_{AA}$ 
\citep{Sasieni:1997,Slager:2001}. 
However, the underlying genetic model is rarely known \textit{a priori}. 
Motivated by the problem of choosing the `right' score vector, 
\citet{Freidlin:2002} proposed the MAX unconditional test as the maximum of three standardised 
CA tests with scores 
$\mathbb{\xi}_\text{dom}, \mathbb{\xi}_\text{add}$, and $\mathbb{\xi}_\text{rec}$
as a global test for association. A similar approach \citep{Zheng:2003,Zheng:2008}
is to introduce a parameter $\eta$ for the score vector 
$\mathbb{\xi}_\eta=(0,\eta,1)$ and to choose $\eta$ in a data-driven way.
This procedure (based on a grid of $\eta$ values) is a special 
case of the general framework described in the following, see the 
\texttt{MAXtest} vignette (Section~\ref{Comp}) for more details.

In the following, we reformulate the test problem and embed the MAX test into a general framework
for conditional inference procedures, derive its limiting 
distribution and propose extensions to stratified designs, 
more than two groups and multiple loci.

\subsection{Reformulation of the Problem}
Let $\Y_i$ denote the case and control status and $\X_i$ the genotype
for all cells $i = 1, \dots, n = 6$. The weights $w_i$ represent the number of 
observations in each cell with total number of observations $N = \sum_i w_i$.
The influence function $h$ provides us with a zero-one 
dummy coding of the groups (being one for cases and zero for controls).
Moreover, three transformations $g$ of the genotype are under test: $g_\text{dom}$ 
assigns scores $\xi_\text{dom}$ to
genotypes $(aa, Aa, AA)$, $g_\text{add}$ assigns scores 
$\xi_\text{add}$ and $g_\text{rec}$
implements scores $\xi_\text{rec}$, cf. Table~\ref{gd}.

\begin{table}
\begin{center}
\caption{Genotype distribution reformulated. \label{gd}}
\vspace*{0.5cm}
\begin{tabular}{lllc|cccc}
$i$ & $\Y_i$  &  $\X_i$  & $w_i$ & $h(\Y_i)$ & $g_\text{add}(\X_i)$ & $g_\text{dom}(\X_i)$ & $g_\text{rec}(\X_i)$ \\ \hline
1 & Case & $aa$ & $r_{aa}$ & 1 & 0 & 0 & 0 \\
2 & Case & $Aa$ & $r_{aA}$ & 1 & 1 & 1 & 0 \\
3 & Case & $AA$ & $r_{AA}$ & 1 & 2 & 1 & 1 \\
4 & Control & $aa$ & $s_{aa}$ & 0 & 0 & 0 & 0 \\
5 & Control & $Aa$ & $s_{aA}$ & 0 & 1 & 1 & 0 \\
6 & Control & $AA$ & $s_{AA}$ & 0 & 2 & 1 & 1 \\
\hline
\end{tabular}

\end{center}
\end{table}

\subsection{Inference Problem and Linear Statistic}

We are interested in testing the null hypothesis of independence of 
grouping $\Y$ and genotype $\X$, i.e., the equality of the distribution $D(\Y)$ 
to the conditional distribution $D(\Y | \X)$ of $\Y$ given $\X$
\begin{eqnarray*}
H_0: D(\Y | \X) = D(\Y)
\end{eqnarray*}
against ordered alternatives. First, we define a three-dimensional
statistic $\T$, each dimension being associated with one of the 
scores $g_\text{add}$, $g_\text{dom}$, and $g_\text{rec}$.
Each statistic is defined by the sum of the scores multiplied by 
the weights associated with cases, i.e., is equivalent to the 
Cochran-Armitage statistic (\ref{CAstat}):
\begin{eqnarray} \label{linstat}
\T = (\T_\text{add}, \T_\text{dom}, \T_\text{rec}) = 
\sum_{i = 1}^n w_i g(\X_i) h(\Y_i) \in \R^{3} 
\end{eqnarray}
with $g(\X_i) = 
(g_\text{add}(\X_i), g_\text{dom}(\X_i), g_\text{rec}(\X_i))$. Thus, the 
three-dimensional linear statistic $\T$ is the vector of the unstandardised
Cochran-Armitage statistics $(\text{CA}(\xi_\text{dom}), \text{CA}(\xi_\text{add}), 
\text{CA}(\xi_\text{rec}))$ for the dominant, additive, and recessive model.

\subsection{Conditional Expectation and Covariance}

The distribution of $\T$  depends on the joint
distribution of $\Y$ and $\X$, which is unknown under almost all practical
circumstances. At least under the null hypothesis one can dispose of this
dependency by fixing the genotypes and conditioning on all possible
permutations of the groups. 
This principle leads to test procedures known as \textit{permutation tests}. 
\citet{StrasserWeber1999} derived closed-form expressions for the conditional 
expectation $\mu \in \R^{pq}$ and covariance $\Sigma \in \R^{3 \times 3}$ 
of $\T$ under $H_0$ given all permutations of the groupings.

The conditional expectation of the influence function $h$ is
\begin{eqnarray*}
\E(h) = N^{-1} \sum_i w_i h(\Y_i) \in \R
\end{eqnarray*}
with corresponding variance
\begin{eqnarray*}
\V(h) = N^{-1} \sum_i w_i \left(h(\Y_i) - \E(h)\right)^2.
\end{eqnarray*}
The conditional expectation of the linear statistic $\T$ is 
\begin{eqnarray}
\mu & = & \E(\T) = \E(h) \sum_{i = 1}^n w_i g(\X_i), \nonumber \\
\Sigma & = & \V(\T) \nonumber \\
& = &
    \frac{N}{N - 1}  \V(h) \times
        \left(\sum_i w_i \left( g(\X_i) g(\X_i)^\top\right) \right)
\label{expectcovar}
\\
& - & \frac{1}{N - 1}  \V(h)  \times \left(
        \sum_i  w_i g(\X_i) \right) \left( \sum_i w_i g(\X_i)\right)^\top.
\nonumber
\end{eqnarray}
The three-dimensional expectation $\mu$ and the three diagonal elements
of the covariance matrix $\Sigma$ contain the mean and the variances
for the additive, dominant and recessive (unstandardised) 
Cochran-Armitage statistics under $H_0$, as given in (\ref{CAstat}) 
and (\ref{linstat}), respectively.

Note that the complete covariance structure, and thus the 
correlation between the elements of the three-dimensional statistic 
$\T$, is known and can be computed for the data at hand.
The corresponding correlation matrix coincides with the correlations
obtained for the three CA test statistics by \citet{Freidlin:2002}.

\subsection{Test Statistics}

Based on the three-dimensional statistic $\T$ and its expectation 
$\mu$ and covariance matrix $\Sigma$, we can easily construct test 
statistics and derive their distribution under the conditions
described in the null hypothesis. As the number of observations 
$N$ tends to infinity, \citet{StrasserWeber1999}
proved that the limiting distribution of the three-dimensional 
statistic $\T$ is a three-dimensional normal distribution with 
expectation $\mu$ and covariance $\Sigma$. Thus, the asymptotic distribution
of a maximum-type statistic
\begin{eqnarray*}
c_\text{max}(\T, \mu, \Sigma)  = \max \left| \frac{\T - \mu}{\text{diag}(\Sigma)^{1/2}} \right|
\end{eqnarray*}
can be evaluated by computing three-dimensional normal probabilities. Here,
$\text{diag}(\Sigma)^{1/2}$ are the conditional standard deviations of
the elements of $\T$. Alternatively, either the sum (or average) statistic
\begin{eqnarray*}
c_\text{sum} = \frac{\mathbf{1}^\top \T - \mathbf{1}^\top\mu}{\mathbf{1}^\top \Sigma \mathbf{1}}, \quad \mathbf{1} = (1, 1, 1)^\top
\end{eqnarray*}
or a quadratic form 
based on the Moore-Penrose inverse $\Sigma^+$ of the conditional covariance
matrix $\Sigma$, i.e.,
\begin{eqnarray*}
c_\text{quad}(\T, \mu, \Sigma)  = (\T - \mu)^\top \Sigma^+ (\T - \mu)
\end{eqnarray*}
can be used and 
follow a standard normal or a $\chi^2$ distribution with two degrees of freedom, respectively. 
The sum and quadratic form statistics, which are competitors for the MAX 
test statistic, reveal high power for an average alternative while the maximum-type 
form for a particular genetic alternative.
In addition, simple linear combinations of the standardised optimum tests for each of
the models have been proposed \citep{Gastwirth:1985}. Often the most appropriate test is not the
average of all these tests but a linear combination of a few ``extreme'' members. Nonlinear
MAX type tests have greater efficiency robustness properties when the minimum
correlation between the optimal tests for the models considered is less than $.70$ 
\citep{Freidlin+Podgor+Gastwirth:1999}, therefore, we focus on the MAX test. A further advantage of 
maximum-type statistics is that information about the likely genetic model underlying the data 
is available using multiplicity-adjusted $p$-values.
Note that, under any circumstances, the exact conditional distribution 
can be approximated by conditional Monte-Carlo methods, which is especially 
attractive for small sample sizes $N$ when we can not expect asymptotics 
to work well.

\subsection{Illustration}

In order to compare the conditional test and its implementation with the
unconditional results reported by \citet{Freidlin:2002}, we reanalyse a study on the association
between a variant of the epidermal growth factor 
(EGF) gene and malignant melanoma according to
\citet[Table~\ref{mel},][]{Shahbazi:2002}.


\begin{table}
\begin{center}
\caption{Melanoma data. \label{mel}}
\vspace*{0.5cm}
\begin{tabular}{l l l l l}
 &  In situ & Control & Total \\ \hline 
AA  &  6 & 32 & 38  \\ 
AG  &  8 & 47 & 55  \\ 
GG  &  10 & 20 & 30  \\ 
Total  &  24 & 99 & 123  \\ \hline
\end{tabular}
\end{center}
\end{table}



The linear statistic $\T$, its conditional expectation $\mu$,
the standard deviations $\sigma = \sqrt{\text{diag}(\Sigma)}$, and 
the corresponding standardised CA statistics are given in
Table~\ref{melres}. 
In addition, we immediately are provided with the covariance matrix
\begin{eqnarray*}
\Sigma = \left( \begin{array}{rrr} 
4.1579 & 5.6255 & 1.4675 \\ 
5.6255 & 10.6845 & 5.0590 \\ 
1.4675 & 5.0590 & 3.5915 \\ \end{array} \right)
\end{eqnarray*}
and corresponding correlation matrix
\begin{eqnarray*}
\text{cor}(\Sigma) = \left( \begin{array}{rrr} 
1.0000 & 0.8440 & 0.3798 \\ 
0.8440 & 1.0000 & 0.8167 \\ 
0.3798 & 0.8167 & 1.0000 \\ \end{array} \right)
\end{eqnarray*}
These values are similar to the correlations between the 
three different CA test statistics as reported by
\citet{Freidlin:2002}.

\begin{table}
\begin{center}
\caption{MAX test for Melanoma data with linear statistic
         $\T$, its conditional expectation $\mu$, standard deviation 
         $\sigma$ and corresponding standardised statistic along with 
         multiplicity-adjusted $p$-values. \label{melres}}
\vspace*{0.5cm}
\begin{tabular}{l l l l l l l l}
 & $\T$ & $\mu$ & $ \sigma$ & $(\T - \mu) / \sigma$ & $p_\text{asympt}$ & $p_\text{step-down}$ \\ \hline
dominant  &  18 & 16.5854 & 2.0391 & 0.6938 & 0.3906 & 0.3302 \\ 
additive  &  28 & 22.4390 & 3.2687 & 1.7013 & 0.0868 & 0.0654 \\ 
recessive  &  10 & 5.8537 & 1.8951 & 2.1879 & 0.0303 & 0.0359 \\ \hline
\end{tabular}
\end{center}
\end{table}


The MAX test has a test statistic equal to 
$2.1879$  and its asymptotic $p$-value 
is 0.0303 (the minimum of $p_\text{asympt}$ in
Table~\ref{melres})
which is roughly the same $p$-value as shown in Table~8 of \citet{Freidlin:2002}.
However, this global $p$-value does not give any information about the
underlying genetic model. Multiplicity-adjusted $p$-values
($p_\text{asympt}$ in Table~\ref{melres})
for each of the dominant, additive, and recessive tests indicate
which mode of inheritance describes the data best
(see Section~\ref{sim} in addition):
These results suggest that the cancer follows a recessive model, although
an additive or intermediate model cannot be ruled out based on these data.

We might want to check whether the asymptotic approximation work well enough
in this situation. The exact conditional
$p$-value is approximated by a conditional Monte-Carlo procedure
with $49999$ random permutations of the data and the corresponding
step-down multiplicity-adjusted $p$-values \citep{WestfallYoung1993}
are given as $p_\text{step-down}$ in Table~\ref{melres}. The small
differences between the asymptotic and approximated $p$-values 
indicates that using the asymptotic distribution is adequate.

\subsection{Generalisations}

A straightforward generalisation is the consideration of $3 \times k$ tables 
instead of $3 \times 2$ tables, where sub-types of cases are compared 
with a control. For example, the genotype distribution
of the control can compare the genotype
of cases with early and late onset of a certain disease. 
A score can be attached to each group, for example 
$1$ to the control group and $-1/2$ for both the early and late onset
cases leading to a linear-by-linear association test. Alternatively,
a trend in the onset of the disease can be described by scores $0, 1, 2$
for the three groups.

In stratified designs, only permutations within each stratum, gender or
family history, are 
admissible; therefore, the expectation $\mu$ and covariance $\Sigma$ has
to be computed separately for each stratum and is then aggregated
over all possible strata. A special version of a stratified design is the commonly used 
meta-analysis, 
including several independent studies as strata, see e.g. \citet{Kavvoura2008}. 
In the absence of a a priori assumption for a particular mode of inheritance, recently 
 \citet{Salanti2008} proposed a Bayesian approach.
 
 In a recent meta-analysis of genome-wide association studies variants on chromosome 9p21.3 
 were identified affecting the coronary artery disease \citep{Schunkert2008} 
 where all seven studies revealed the same additive mode of inheritance using the here proposed approach.


Finally, it is interesting to consider multiple loci, i.e., 
multiple genotype distributions, simultaneously. For two loci, 
we can look at all six CA tests by defining a linear statistic
$\T$ containing the three CA tests for the first as well 
as the three CA tests for the second locus. As a consequence,
we can compute the complete covariance matrix and take the 
underlying correlations between the two loci as well as between the three genetic models into account.

\section{Illustration and Application}

\begin{table}
\begin{center}
\caption{Psoriasis data \label{reich}}
\vspace*{0.5cm}
\begin{tabular}{llllll} \\
\multicolumn{6}{c}{IL1B\_511 locus} \\
\multicolumn{6}{c}{} \\ 
Male  &  &  Control & Early Onset & Late Onset & Total \\ \hline 
 & CC  &  75 & 54 & 29 & 158  \\ 
 & CT  &  93 & 44 & 13 & 150  \\ 
 & TT  &  14 & 7 & 4 & 25  \\ 
 & Total  &  182 & 105 & 46 & 333  \\ 
Female  &  &  Control & Early Onset & Late Onset & Total \\ \hline 
 & CC  &  76 & 26 & 17 & 119  \\ 
 & CT  &  69 & 20 & 10 & 99  \\ 
 & TT  &  18 & 5 & 2 & 25  \\ 
 & Total  &  163 & 51 & 29 & 243  \\ \multicolumn{6}{c}{} \\
\multicolumn{6}{c}{TNFA\_238 locus} \\
\multicolumn{6}{c}{} \\
Male  &  &  Control & Early Onset & Late Onset & Total \\ \hline 
 & GG  &  170 & 71 & 40 & 281  \\ 
 & GA  &  12 & 33 & 6 & 51  \\ 
 & AA  &  0 & 1 & 0 & 1  \\ 
 & Total  &  182 & 105 & 46 & 333  \\ 
Female  &  &  Control & Early Onset & Late Onset & Total \\ \hline 
 & GG  &  146 & 43 & 24 & 213  \\ 
 & GA  &  17 & 8 & 5 & 30  \\ 
 & AA  &  0 & 0 & 0 & 0  \\ 
 & Total  &  163 & 51 & 29 & 243  \\ \hline
\end{tabular}
\end{center}
\end{table}


\citet{Reich:2002} investigate the association between psoriasis and 
polymorphisms of genes encoding tumour necrosis Factor-$\alpha$ and 
Interleukin-$1\beta$ where for the IL1B\_511 locus the related 
$3 \times 2$ table data are given in Table~\ref{reich}.
A control group and two groups of affected people with early and
late onset of the disease are under test. One is
interested in detecting any deviation from independence of
genotype distribution for both loci and the three groups in either
females and / or males. Attaching scores $1, -1/2, -1/2$ to the
control, early and late onset group results in a linear statistic
$\T$ with six elements: three models for each of the two loci.


The multiplicity-adjusted $p$-values in Table~\ref{pstab} indicate that
there is a strong deviation from independence for the TNFA\_238 locus.
The recessive model has the largest $p$-value and thus it is not likely
that this model is true. The $p$-values for the dominant and the 
additive are extremely small, so either of these models could have generated 
the data. We can simultaneously reject the null hypothesis of independence
between the genotype distribution of the IL1B\_511 locus and the three groups.
Here, the dominant model seems to explain the data best. 

Our analysis improves upon the original analysis of these data by \citet{Reich:2002}
with respect to three points: All three groups and the stratification
by gender are taken into account and and the new test makes use of the
correlation between the two loci instead of applying a Bonferroni correction
in order to maintain an overall significance level.


\begin{table}
\begin{center}
\caption{MAX test for psoriasis data: Asymptotic multiplicity-adjusted $p$-values. \label{pstab}}
\vspace*{0.5cm}
\begin{tabular}{lrr}
 & TNFA\_238 & IL1B\_511 \\ \hline
dominant  &  $< 0.0001$&0.0407 \\ 
additive  &  $< 0.0001$&0.1051 \\ 
recessive  &  0.7241&0.9819 \\ \hline
\end{tabular}
\end{center}
\end{table}

Recently, \citet{Bagos2007} proposed a penalty-free selection approach for the underlying mode of inheritance
based on the parameter estimates in a logistic regression model. Here, we re-analyse their meta-analysis 
data on the association of KIR6.2 gene polymorphism with type II diabetes 
\citep[Table~1 on page 3 in][]{Bagos2007}.



\begin{table}
\begin{center}
\caption{MAX test for Type II diabetes data  with linear statistic  
         $\T$, its conditional expectation $\mu$, standard deviation 
         $\sigma$ and corresponding standardised statistic along with 
         multiplicity-adjusted $p$-values. \label{diares}}
\vspace*{0.5cm}
\begin{tabular}{l l l l l l l l}
 & $\T$ & $\mu$ & $ \sigma$ & $(\T - \mu) / \sigma$ & $p_\text{asympt}$ & $p_\text{step-down}$ \\ \hline
dominant  &  337 & 320.4287 & 7.0531 & 2.3495 & 0.0208 & 0.0112 \\ 
additive  &  438 & 404.0265 & 10.1618 & 3.3433 & 0.001 & 0.001 \\ 
recessive  &  101 & 83.5979 & 5.2912 & 3.2889 & 0.0014 & 0.0013 \\ \hline
\end{tabular}
\end{center}
\end{table}

\section{Simulation Experiments \label{sim}}

It might be questioned if the minimal $p$-value can be observed for the
correct mode of inheritance with high probability and thus how good the `estimator' 
is under practical circumstances. The frequency of
correct model identifications and the power of the MAX test is investigated
in some simple situations in the following.

Many different patterns of penetrances $f_j$, $j \in \{aa, aA, AA\}$, 
disease prevalence $p$, sample size of cases and controls 
$R$, $S$ can be investigated in a simulation study. We will 
focus on a high prevalent disease (i.e., $p=0.5$), penetrances according 
to an additive, recessive and dominant genetic model (as well as no 
association characterising the null hypothesis) for a total sample size 
of $N=400$ divided into the balanced $R=S=200$ and several unbalanced sampling schemes. 
Unbalanced data are of interest because real data examples exist with 
seriously higher control sample size, see e.g. the data in Table~\ref{mel}, 
or with more cases, see e.g. the IL13 polymorphism in atopic 
dermatitis \cite{Neuhauser:2002}, Table 4. For the proposed MAX test both the global 
power $\pi_\text{global}$(the decision rate in favour of any alternative) and the 
correct model identification rates $\psi_\text{add}$,$\psi_\text{rec}$,$\psi_\text{dom}$ 
are compared with the power of the individual genetic model tests 
$\pi_\text{add}$, $\pi_\text{rec}$, $\pi_\text{dom}$ in Table~\ref{simtab}.

\begin{table}
\begin{center}
\caption{Type I error rate and empirical power estimates ($\pi$) for prevalence $p=0.5$ based on 10000 runs
         with sample sizes $R$ (cases) and $S$ (controls) along with correct model identification rates
         $\psi$. \label{simtab}}
\vspace*{0.5cm}
  \begin{tabular}{l l l l| l l l |l l l}
        Model & R & S & $\pi_{\text{global}}$ &  $\psi_\text{add}$ & $\psi_\text{rec}$ & $\psi_\text{dom}$ & $\pi_\text{add}$ & $\pi_\text{rec}$ & $\pi_\text{dom}$   \\
 \hline
 Null & 200 & 200 & 0.048&  0.012 & 0.017&  0.019           &   0.051 & 0.047 & 0.049 \\
 dominant  & 200 & 200 & 0.85 &  0.13 &  0.01 &  \textit{0.71} & 0.75 &  0.23 &  \textbf{0.91} \\
 additive  & 200 & 200 & 0.84 &  \textit{0.53} & 0.10  & 0.21 &  \textbf{0.88} & 0.71 &  0.78 \\
 recessive  & 200 & 200 & 0.86 &  0.16 &  \textit{0.69} & 0.01 &  0.80 &  \textbf{0.91} & 0.28 \\
 dominant  & 100 & 300 & 0.72 &  0.16 &  0.01 &  \textit{0.55}&  0.63 &  0.21 & \textbf{0.80} \\
 additive  & 100 & 300 & 0.73 &  \textit{0.43} & 0.14 &  0.16 &  \textbf{0.78} & 0.60 &  0.65 \\
 recessive  & 100 & 300 & 0.77 &  0.16 &  \textit{0.60} & 0.01 &  0.67 &  \textbf{0.82} & 0.22  \\
 dominant  & 300 & 100 & 0.76 &  0.11 &  0.01 &  \textit{0.64} & 0.66 &  0.19 &  \textbf{0.82} \\
 additive  & 300 & 100 & 0.75 &  \textit{0.42} & 0.09 &  0.24 &  \textbf{0.79} & 0.59 &  0.69 \\
 recessive  & 300 & 100 & 0.75 &  0.17 &  \textit{0.55} & 0.01 &  0.69 &  \textbf{0.82 }& 0.24 \\
\hline
 \end{tabular}
\end{center}
 \end{table} 

 
Per definition all tests control the type I error rates. Clearly, the power is maximal for 
the individual, unadjusted tests when the genetic model is known (bold marked). But the a 
priori knowledge of the genetic model is commonly unrealistic. For balanced samples sizes the power 
of the MAX test is independent of the underlying genetic model and marginal smaller 
compared with the maximum power for the known model. Additionally to the global decision 
that a significant association exists, the MAX test provide an adjusted $p$-value for the 
most likely genetic model. In this case, the identification of the additive model is most difficult 
because of the two equal competitors; whereas the identification of the dominant or recessive 
model is easier because the additive model is the only competitor. For unbalanced 
designs the power decreases although the total sample size remains constant.

\section{Computational Details \label{Comp}}

The \Rpackage{coin} add-on package \citep{Hothorn:2006:AmStat, Hothorn+Hornik+VanDeWiel:2008, PKG:coin}
to the \textsf{R} system for statistical computing \citep{rcore2007}
provides an implementation of the conditional inference framework
sketched in this section. The analysis of an association study 
by the MAX test only requires the user to set-up the score function $g$. Then, the
function \texttt{independence\_test} can be used to perform 
the MAX test and to compute multiplicity-adjusted $p$-values.
All analyses presented in this paper
are reproducible by means of the \texttt{MAXtest} package vignette
accessible from within \textsf{R} via
\begin{Schunk}
\begin{Sinput}
> vignette("MAXtest", package = "coin")
\end{Sinput}
\end{Schunk}
This dynamic document also includes the simulation source code.

\section{Conclusions}

We propose a flexible approach to permutation tests for association of a    
bi-allelic marker with a disease in population-based case-control 
studies. The joint conditional asymptotic distribution of the 
three underlying 
linear association tests, i.e., Cochran-Armitage tests with optimal
scores for additive, dominant, and recessive modes of inheritance,
is known and can be used to approximate the distribution of the 
corresponding maximum statistic.
Not only a global $p$-value can be derived this way but also
multiplicity-adjusted $p$-values for each of the individual models. 
When the mode of inheritance is unknown, remarkably high correct model 
selection rates can be achieved. Based on a general framework for conditional 
inference we extend the MAX test to stratified designs, $3 \times k$ tables 
as well as multiple loci. Correlations between 
loci and corresponding association tests are taken into account leading to 
more powerful multiple test procedure. Further modifications and extensions
and specific choices of $g$ and $h$ are described by \cite{Hothorn:2006:AmStat} 
and \cite{Hothorn+Hornik+VanDeWiel:2008}.
For small sample sizes, a better approximation of the $p$-values 
can be obtained from Monte Carlo resampling techniques.
For genome-wide association studies, computing time is a limiting factor; 
a further advantage for the above approach \citep{Ziegler2008}.

The proposed procedures are easily 
applicable using the computational tools provided by the \textsf{R} 
add-on package \Rpackage{coin} as illustrated in a dedicated package vignette. A future modification will be 
the use of model-specific genomic-control corrected tests 
analogously to \citet{Zang:2007} in the possible case of 
population stratification.

\section{Acknowledgements} 
We would like to thank Andreas Ziegler and Inke R. Koenig for providing 
us with the psoriasis data.

\bibliographystyle{bimj}
\bibliography{references}

\end{document}

\newpage

\section*{Appendix}

The Melanoma data are represented by a \texttt{table} object in \textsf{R}
as follows:
\begin{Schunk}
\begin{Sinput}
> me <- as.table(matrix(c( 6,  8, 10,
+                32, 47, 20), byrow = TRUE, nrow = 2,
+     dimnames = list(Group = c("In situ", "Control"),
+                     Genotype = c("AA", "AG", "GG"))))
> me <- t(me)
> me
\end{Sinput}
\begin{Soutput}
        Group
Genotype In situ Control
      AA       6      32
      AG       8      47
      GG      10      20
\end{Soutput}
\end{Schunk}
The function $g$ is implemented by the following function:
\begin{Schunk}
\begin{Sinput}
> add <- c(0, 1, 2)
> dom <- c(0, 1, 1)
> rec <- c(0, 0, 1)
> g <- function(x) {
+     x <- unlist(x)
+     cbind(dominant = dom[x], additive = add[x], recessive = rec[x])
+ }
\end{Sinput}
\end{Schunk}
which then sets up the MAX test for the Melanoma data:
\begin{Schunk}
\begin{Sinput}
> library("coin")
> it <- independence_test(me, xtrafo = g, alternative = "greater")
> it
\end{Sinput}
\begin{Soutput}
	Asymptotic General Independence Test

data:  Group by Genotype (AA, AG, GG) 
maxT = 2.1879, p-value = 0.03042
\end{Soutput}
\end{Schunk}
The multiplicity-adjusted $p$-values for both inference and
estimating the underlying mode of inheritance are computed 
via:
\begin{Schunk}
\begin{Sinput}
> pvalue(it, method = "single-step")
\end{Sinput}
\end{Schunk}
\begin{Schunk}
\begin{Soutput}
  dominant   additive  recessive 
0.39063104 0.08676118 0.03038007 
\end{Soutput}
\end{Schunk}

The score independent approach by \cite{Zheng:2008} maximises the
Cochran-Armitage statistic over potential score vectors 
$\xi_\eta = (0, \eta, 1)$, i.e., one is interested in the test statistic
$c_\text{max}$ where the maximium is taken over a grid of $\eta$ values.
This procedure can be performed by choosing appropriate transformation
$g$, i.e., a grid of $\eta$ values in $[0,1]$
\begin{Schunk}
\begin{Sinput}
> gZheng <- function(x) {
+     x <- unlist(x)
+     eta <- seq(from = 0, to = 1, by = 0.01)
+     tr <- sapply(eta, function(n) c(0, n, 1)[x])
+     colnames(tr) <- paste("eta", eta, sep = "_")
+     tr
+ }
> itZ <- independence_test(me, xtrafo = gZheng, alternative = "greater")
> itZ
\end{Sinput}
\begin{Soutput}
	Asymptotic General Independence Test

data:  Group by Genotype (AA, AG, GG) 
maxT = 2.1879, p-value = 0.03151
\end{Soutput}
\end{Schunk}
\begin{Schunk}
\begin{Sinput}
> pvalue(itZ, method = "single-step")
\end{Sinput}
\end{Schunk}
\begin{Schunk}
\begin{Soutput}
     eta_0   eta_0.01   eta_0.02   eta_0.03   eta_0.04   eta_0.05   eta_0.06 
0.03151286 0.03162379 0.03176212 0.03191972 0.03209197 0.03228663 0.03250561 
  eta_0.07   eta_0.08   eta_0.09    eta_0.1   eta_0.11   eta_0.12   eta_0.13 
0.03274823 0.03301281 0.03330016 0.03361374 0.03395610 0.03432383 0.03472546 
  eta_0.14   eta_0.15   eta_0.16   eta_0.17   eta_0.18   eta_0.19    eta_0.2 
0.03515716 0.03561464 0.03610973 0.03663956 0.03720648 0.03781138 0.03845817 
  eta_0.21   eta_0.22   eta_0.23   eta_0.24   eta_0.25   eta_0.26   eta_0.27 
0.03914781 0.03987640 0.04065208 0.04147678 0.04235428 0.04328269 0.04426266 
  eta_0.28   eta_0.29    eta_0.3   eta_0.31   eta_0.32   eta_0.33   eta_0.34 
0.04530451 0.04640300 0.04756671 0.04879473 0.05009090 0.05145841 0.05289975 
  eta_0.35   eta_0.36   eta_0.37   eta_0.38   eta_0.39    eta_0.4   eta_0.41 
0.05441897 0.05601775 0.05770110 0.05946786 0.06132908 0.06327947 0.06533041 
  eta_0.42   eta_0.43   eta_0.44   eta_0.45   eta_0.46   eta_0.47   eta_0.48 
0.06747833 0.06973053 0.07209088 0.07456038 0.07714360 0.07984304 0.08266065 
  eta_0.49    eta_0.5   eta_0.51   eta_0.52   eta_0.53   eta_0.54   eta_0.55 
0.08560310 0.08867050 0.09186635 0.09519253 0.09865287 0.10224856 0.10598309 
  eta_0.56   eta_0.57   eta_0.58   eta_0.59    eta_0.6   eta_0.61   eta_0.62 
0.10985423 0.11386935 0.11802337 0.12232122 0.12676362 0.13134787 0.13608037 
  eta_0.63   eta_0.64   eta_0.65   eta_0.66   eta_0.67   eta_0.68   eta_0.69 
0.14095192 0.14596884 0.15112700 0.15642556 0.16186212 0.16743629 0.17314356 
   eta_0.7   eta_0.71   eta_0.72   eta_0.73   eta_0.74   eta_0.75   eta_0.76 
0.17898205 0.18494849 0.19104191 0.19725298 0.20358348 0.21002529 0.21657605 
  eta_0.77   eta_0.78   eta_0.79    eta_0.8   eta_0.81   eta_0.82   eta_0.83 
0.22322887 0.22998047 0.23682421 0.24375584 0.25076891 0.25785708 0.26501653 
  eta_0.84   eta_0.85   eta_0.86   eta_0.87   eta_0.88   eta_0.89    eta_0.9 
0.27223991 0.27952138 0.28685492 0.29423443 0.30165464 0.30910861 0.31659023 
  eta_0.91   eta_0.92   eta_0.93   eta_0.94   eta_0.95   eta_0.96   eta_0.97 
0.32409437 0.33161541 0.33914685 0.34668311 0.35421895 0.36174932 0.36926863 
  eta_0.98   eta_0.99      eta_1 
0.37677208 0.38425491 0.39171216 
\end{Soutput}
\end{Schunk}
The test statistic takes its maximum for the score $(0, 0, 1)$ 
(as above), its corresponding adjusted $p$-value is only marginally larger compared
to the MAX test based on three score vectors (0.03151
vs. 0.03042).

\end{document}


