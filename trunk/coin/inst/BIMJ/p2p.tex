
\documentclass{article}
\usepackage{amstext}
\usepackage{natbib}

\title{Point to Point Answers for Manuscript BIMJ-2008-00203}

\begin{document}

\maketitle

We would like to thank both referees and the associate editor for
their helpful comments on our manscript ``Order-restricted Scores Test for
the Evaluation of Population-based Case-control Studies when the
Genetic Model is Unknown''. The manuscript has been revised taking all 
the comments and suggestions into account; we comment on details
in the following.

\section*{Referee I}

\begin{itemize}
\item \textit{The author reformulated MAX into a different frame work. 
However, the benefits of this effort have not been clearly shown. It is 
known the distribution of MAX can be obtained by multiple integrations. 
In addition, no new insights are obtained through the reformulation.
For improvement, the authors need to show if the calculation is easier 
with this new reformulation? Or any new insights can be obtained?}

Our paper solves the technical difficulties that make
the application of the MAX test as described by 
\cite{Freidlin:2002} burdensome. In addition, our theory
suggests straightforward extentions, such as
tests for stratified designs, designs with more than two
groups and two or more loci within one framework. This is not possible
in an unconditional world. The abstract and introduction have been rewritten
to express these points more prominently.

\item
\textit{Page 2 two lines above eqn (1): The statement of alternative was not 
correct. Because sume of p's and q's are 1. From H1, one would have 
$1 =p_{aa}+p_{aA}+p_{AA} < q_{aa}+q_{aA}+q_{AA}=1$. 
The order should be stated in terms of risks, e.g., $P(\text{disease}|\text{a
genotype})$.}

Thank you very much for this hint. The alternative is now formulated
in terms of penetrances.

\item
\textit{Page 4 (last equation), $c_\text{quad}$ is the same as Pearson's test?}.

No, $c_\text{quad}$ is a quadratic form collapsing three correlated 
CA-statistics.

\item
\textit{Page 5 (first line in sec 2.5): "wan" should be "want".}

Thanks, fixed.
\end{itemize}


\section*{Referee II}
\begin{itemize}
\item
\textit{
There is a need to clarify the connection between the results derived 
by Strasser and Weber (1999) and the MAX test statistic. This is 
important because the approximation of the distribution of the MAX 
test statistic was based on the work by Strasser and Weber (1999).}

The MAX test is a special case of the linear statistics studied
by \cite{StrasserWeber1999}. The introduction and technical parts
of the paper were extended to make this point clear.

\item
\textit{Page 4, authors need to notation in two formulas in section 2.4, 
such as the meaning of $\text{diag}(\Sigma)^{1/2}$ and $\Sigma^+$.}

Done.

\item
\textit{
We will focus on a high prevalent disease (i.e., $p=0.5$). 
What is the rational for such consideration?}

For the sake of comparability we choose a simulation
design similar to previously published one, e.g., in
\cite{Neuhauser:2002}, which investigated the high prevalent 
case as well.

\end{itemize}

\section*{AE}

\begin{itemize}
\item 
\textit{
The paper addresses an interesting question and the authors have a very good
research record; however, the referees’ reports were lukewarm at best. A major criticism,
which the AE agrees with, is that the relationship of the procedure to previous work is
unclear. In particular, one reviewer noted the need to explain the connection of the results
in Strasser and Weber (1999). The other reviewer was more negative questioning whether
any new insights were given in the paper. Part of the problem is that the authors do not
integrate their results into the existing literature on efficiency robust tests. Indeed, neither
robust test nor efficiency robustness is listed in the Key words. In the latest issue of
Biometrics an article on trend tests by Zheng (2008) notes that the relationship between
the correlation matrix of the “best tests” for various models, reported in the example in
Section 2.5, was used to develop robust tests for the two-sample problem by Gastwirth
(1966 JASA). Of perhaps greater relevance to the use of the method on stratified data in
Section 3 is a subsequent paper by Gastwirth (1985 JASA), also noted by Zheng (2008)
that developed simpler efficiency robust tests for stratified dose response data, which are
similar to the data the authors analyze. Another article by Podgor and Mehta (1996
Statistics in Medicine) developed efficiency robust tests for ordered R x C tables. For
each locus the data in Table 5 are ordered 3x3 tables. These earlier articles did not focus
on the MAX but used a robust linear combination of the optimum tests that depended on
the correlation structure. Later, Freidlin et al. (1999 Biometrics) showed that when the
minimum correlation of the best tests was at least .70, the robust linear combination was
fine but when the minimum correlation was .60 or less the MAX should be used. This
background material will assist the reader in understanding why the authors are using a
rather computationally intensive approach.}

We thank the AE for these insightful comments. In fact, the interesting
paper by \cite{Zheng:2008} was published after the submission of our manuscript
to Biometrical Journal. For case-control studies, where the set of
possible scores is $(0, \eta, 1)$, the procedure by
\cite{Zheng:2003,Zheng:2008}
can be implemented using the theory presented in our paper as well.
Two actions were taken: 1) We describe the connection of several relevant MAX
tests proposed in the literature in the introduction more carefully 
and 2) the Appendix has been extended analysing the Melanoma data with
the MAX test by \cite{Zheng:2008}. However, we think that the extensive literature
on the usage of max-type test statistics for robust analysis of 
ordinal data doesn't add something to the simple MAX statistic for three 
a priori known score vectors. In fact, the computations associated
with most of the procedures mentioned above are burdensome (as the recent
paper by Zheng indicates). In contrast,
the conditional view on the problem adopted in our manuscript provides
us with computationally simple (i.e., well-studied) solutions.

\item
\textit{
On page 2, in section 1, the authors claim that the MAX is an intuitive approach. An
even more intuitive compromise test would seem to be the AVERAGE. Of course, when
there are more than two plausible underlying models we know from the 1966 article that
this may not even be the most robust linear test.}

Thank you very much for this hint. We extended the description of possible
test statistics to the sum (or average) statistic, which can be conveniently
be described using the techniques mentioned in our manuscript.

\item
\textit{On page 3, the use of the term influence function should be avoided as in the
robustness area the influence function (introduced by Huber and developed further by
Hampel and others) has an entirely different meaning and use.}

The term influence function is used in a more general sense here,
allowing for both robust and non-robust influence functions $h$, see
\cite{StrasserWeber1999} for the technical details.

\item 
\textit{Basically you are using the conditional distribution rather than the unconditional
distribution to conduct your analysis. This is similar to using the Mantel-Haenszel (1959)
method of combining 2x2 tables (using hypergeometric variance for each table) instead
of Cochran (1954) binomial variances. Asymptotically, they become equivalent and your
correlations and tests are also very close to the Freidlin et al. (2002)
paper.}

Yes, but that's a feature. By the same arguments, the two-sample $t$ test and
its permutation version are asymptotically equivalent because the
$t$ distribution tends to the normal distribution with $n \rightarrow
\infty$.

\item
\textit{The AE questions whether treating each of the case groups (early and late onset)
equally as the weighting 1, -.5, -.5 used is the most appropriate approach as genetically
caused diseases tend to occur at younger ages. Thus, the data in Table 5 might be ordered
as control, late onset and early onset. What effect will this have on the analysis? Is the
relationship more “significant”? This is quite important as there are two possible models
here: a) combining ordered 3x3 tables OR b) combining tables testing two cases vs. one
control. Both approaches lead to correlated optimum tests etc. but the correlation matrices
and hence best efficiency robust test may be different. This issue should be developed
further because the previous efficiency robustness approach has not been used on
stratified r x c tables; although it has been used on stratified dose response data. Also, the
multiple cases or multiple control group contexts would also be somewhat new. If the
authors could expand the analysis here, they would respond to the criticism of one of the
referee’s that the results aren’t really an advance.}

We agree that the choice of contrasts for this example is debatable.
In fact, stratified $3 \times k$ tables can be handled by appropriate
choices of the influence function $h$ (as outlined in the introduction,
section 5 and the dicussion). We hestitate to expand this issue further
since this goes far beyond the scope of the paper. We refer to two
other publications which describe the richness of the Strasser-Weber
framework in more detail.

\item
\textit{
The authors discuss correlations in the data for tables at different loci. This is an
interesting issue but needs more discussion. If the loci are on different loci under the null
hypothesis of no association of either locus with the disease, the data should not be
correlated. If the loci are near each other on the same chromosome the data may well be
correlated under the null. Some discussion of the underlying genetics is needed here in
order to justify the claims made.}

The multiple testing procedure involved adapts itself to the correlation
of the two response variables and maintains it size whether or not the
two loci are correlated.

\item
\textit{
The authors should give the data used in Table 7, rather than ask the reader to locate
the original paper. Actually, this example does not seem to add much as the tests for all
three models reject the null hypothesis. This may distract the reader from the main point
of the article, the development of a conditional MAX efficiency robust test and its use on
stratified data. Replacing this example by an expanded discussion of the meta-analysis
example would make the paper more interesting.}

The paper is reproducible from the \textit{MAXtest} package vignette which
contains all data and analyses (freely available from
\texttt{CRAN.R-project.org}), there is no need to actually print
tables or simulation code.

\item
\textit{
On page 8, last paragraph of Sec. 4—what does “non-inferior smaller compared with”
mean?}

The sentence was misleading and has been rephrased.

\item
\textit{
In Sec. 6, are multiple loci the same as multiple end points in clinical trials?}

The sentence has been clarified.

\item
\textit{
In Sec. 6, the claim that the p-values are better approximated by re-sampling techniques
may well be correct but it requires far more substantiation than provided in the current
text.
}

The \emph{exact} $p$-value can be approximated up to any desired accuracy
by conditional Monte-Carlo methods, regardless of the sample size. 
The limiting distribution is adequate only for large sample sizes, as
explained in Section 2.5.

\end{itemize}

\bibliographystyle{plainnat}
\bibliography{references}


\end{document}