
\documentclass[11pt]{article}
\usepackage{hyperref}
\usepackage{amstext}
\usepackage{amsfonts}
\usepackage{graphicx}
\usepackage[round]{natbib}

\renewcommand{\baselinestretch}{1.2}

\begin{document}

\title{Point to Point Answers for Manuscript \\
`A Lego System for Conditional Inference' \\
TAS MS05-239, Reviewer \#1}
\author{}
\maketitle

\begin{enumerate}
\item \textsl{The ideas raised in this paper are interesting. However, the presentation of the 
              material lacks sufficient explanations for the interested or committed reader. 
              Many of the sentences have compound ideas that beg for further elaboration. 
              This is especially important because the two main sources (Strasser and Weber, 1999; 
              Janssen and Pauls, 2003) are not physically or intellectually accessible to the 
              typical reader (Mathematical Methods of Statistics is a Russian Journal, and I have 
              had to order this background paper from my library). The subject-matter examples are 
              missing important details and the numerical results are not always explained in the 
              original context. It is often not clear what one gain with the methods presented here 
              that would not be obtained with standard methods.}


The comments by referee \#1 have been extremely helpful for us when
identifying shortcomings in both the description of the theoretical
building blocks as well as the applications. We refer to a list of modifications
as indicated in the point to point answers to the specific comments below.

Indeed, the original publication is difficult to access, we therefore added a link
to a preprint to the list of references.


\item \textsl{The documentation for the new R-procedure \texttt{independence\_test()} is also just not 
              adequate. The paper needs to state exactly what each input and output from the 
              software means. Also, do not assume that the typical reader will be willing to 
              read through S-type or R-type code or output without careful guidance. Most will 
              simply gloss over the code or output.}

We thank the referee for pointing our places with insufficient documentation
of the procedures applied in this manuscript. We hope that the modifications
and extensions made to the corresponding paragraphs enhance the readability
for the non-\textsf{S} user (please see below for specific
changes).

\item \textsl{As a side point, is this R-procedure appropriately named, or should it be called 
              something else that reflects the conditional nature of the inference?}

The name of the package (\texttt{coin}) was chosen in order to reflect that
all methods implemented are conditional by nature. The names of the specific
functions either come from the corresponding classical test
(\texttt{kruskal\_test}) or the test problem addressed (most general:
\texttt{independence\_test}).

\item \textsl{It should also be noted that 'Lego' is a registered trademark and should not be used 
              as an adjective to describe a procedure. Furthermore, there is nothing about the 
              proposed procedure that makes one think specifically of Lego building blocks, 
              let alone any other building blocks (beyond the most metaphorical sense). The 
              metaphor of the 'spanner' also seems stretched.}

We checked on the legal issue prior to submitting the manuscript to TAS. Our
reading of \url{http://www.lego.com/eng/info/default.asp?page=fairplay} is
that the use of the word `Lego' (which is a trademark) is admissible for our
purposes. Moreover, there are other scientific publications which used the
term `Lego' for building block concepts, for example
\url{http://ieeexplore.ieee.org/xpl/abs_free.jsp?arNumber=129414}.
Referee \#2 seems to like the spanner
metaphor, so we decided to keep this formulation as a bridge between
introduction and summary.


\item \textsl{page 1:  What is meant by "non-standard inference procedures" here? The classical 
              approaches seem pretty standard.}

This formulation was misleading, we wanted to say that the data analysis
problems are non-standard (ordered alternatives, multiple responses etc.).
This has been clarified.

\item \textsl{page 2:  The introduction needs to be much clearer. Make it clear what is meant by 
              "certain null hypotheses." For example, there are null hypotheses that specify the 
              underlying cell proportions in a contingency table, and there are null hypotheses 
              that specify only independence between rows and columns, but not the proportions. 
              Also, not all conditional tests are permutation tests; there are both conditional 
              exact permutation tests and conditional asymptotic tests. Also, mention some of 
              the pros and cons of permutation tests explicitly here.}

The first sentence refers to statistical test procedures in general,
without having a special test problem or null hypothesis in mind. We removed
`certain' here. In this paper, the null hypotheses are always defined in
terms of independence without an explicit formulation of the alternative
(1st paragraph in Section 2).

We refer to permutation tests as to inference procedures conditioning on all
\marginpar{wo liegt das problem?}
permutations of the data, regardless of how the null distribution of a test
statistic is computed or approximated. Therefore, a test which is obtained
from a normal approximation of the null distribution (which itself is
`exact' by definition) is again a permutation test. A clarification has been
added to the second paragraph in Section 2 and we now use the keyword 
`Asymptotic distribution' instead of `Approximations'.

Our impression is that a rather long debate on pros and cons of permutation
tests (the main contributions are cited in the introduction) lead to standoff
between the `pro' community and statisticians from a more parametric
modeling school. Our interest in this research clearly induces a bias
towards arguments from the first community and we hesitate to make
statements about the usefulness of permutation tests in general or to
re-state facts and opinions which have been made in the past.

\item \textsl{page 3:  The Wilcoxon-Mann-Whitney and Cochran-Mantel-Haenszel tests are not 
              so much inflexible as specific to particular problems. It is indeed nice that they 
              can be generalized, but this in no way detracts from their value or utility, where 
              appropriate. Conversely, generalized methods lack some of the specificity of 
              simpler methods and become hard to interpret.}

The term `specific' is much more adequate to express our meanings here,
thank your very much for this suggestion! 

\item \textsl{page 4:  Elaborate on "permutation symmetric way" here, meaning the value of the 
              function does not depend on the order in which the Y-variables appear.}

Yes, we added an explanation on `permutation
symmetry' to this paragraph.
 
\item \label{asympt} \textsl{page 5:  Regarding the last sentence, "Less well known ..." It is not obvious that 
              the conditional distribution can be approximated by its limiting distribution 
              under - all - circumstances, but in any case the accuracy of this approximation 
              is what is of most concern. The implications of this and the theory in the rest of 
              this paragraph should be developed more clearly. }

The first sentence in this paragraph was misleading, thank you very much for
pointing this out. We wanted to say that it is \textit{possible} to
approximate the conditional distribution using asymptotic arguments but did
not comment on the quality of this approximation, which of course depends on
the data at hand. We reformulated this sentence.

\item \textsl{pages 6-8:  In the "genetic components of alcoholism" example, some essential details are 
              missing from the text, such as the sample size in each group, that there are three allele 
              lengths (short, intermediate, long), and how this relates to the case-control study from 
              which the data were taken, etc. Some of these details can be figured out from the figure 
              and subsequent text, but the logical flow is awkward.  As for the R-code, it would help 
              to include explicitly the formula statement: "formula = elevel $\sim$ alevel". The "trafo" 
              statement is not explained either.  As for the alternative approach, why were these 
              particular scores (2, 7, 11) chosen, other than that they were chosen before? 
              These choices seem to be data driven. 
              Using the Jonkheere-Terpstra test seems to be a better way to get at the ordering of 
              these allele levels. Can this test be implemented in the new procedure? 
              Finally, it is not advisable to compare methods based on p-values, since they measure 
              the statistical significance of effects with respect to particular hypotheses, but 
              ignore other factors, like absolute magnitude, that may also be important. After all, 
              both p-values are pretty small.}

The above comments have been most valuable for us in order to come up with
a description of the practical steps which should now be more easily
accessible for non-\textsf{S}
users. The study design and the experimental setup are now described in more
detail. The paragraph explaining the arguments of the
\texttt{independence\_test} function has been restructured. We now start with the 
conventional function \texttt{kruskal.test} which at least the \textsf{S} users
are familiar with. Then, the additional arguments required for \texttt{independence\_test}
are explained in more detail.

The scores for the allele length variable are
obtained from the discrete variable underlying the allele length measurements
and are thus not data driven. The Jonckheere-Terpstra test 
can be re-written in
terms of Helmert-contrasts, however, the pairwise ranking (instead of a
joint ranking) induces problems whose solution would exceed the scope of
this manuscript.

We didn't mean to compare two tests by means of $p$-values but wanted to
indicate that using a test which takes the ordering information into account
seems to be appropriate. The formulation has been changed.

\item \textsl{page 9:  In the "smoking and Alzheimer's disease" example, the paper assumes that the 
              readers will recognize the nesting structure (within gender) in the R-code. Also, 
              why did the "ytrafo" and "trafo" statements disappear? }
 
We added a note on gender to the first paragraph. The default values for
\texttt{xtrafo} and \texttt{ytrafo} are explained in more detail now.

\item \textsl{page 10:  It would help to state in plain English that the R-code restricts the 
              analysis to a particular subset (males, females).}
 
fixed.

\item \textsl{page 11:  The paper needs to explain the calculation of the quantiles of the 
              permutation distribution better. Why is the 95\% quantile (2.81) so large compared to the 
              usual two-sided normal-score of 1.96 (and is this the correct comparison)? Also, 
              the paper needs to explain how the p-values were adjusted for multiple testing 
              (Bonferroni?).}

The quantile is obtained from the $16$-dimensional limiting normal distribution
of the contingency table $T$ and thus $p$-values can be adjusted by a
max-T multiple testing approach (since the complete covariance matrix is
known in our case). This has been clarified in the paper, however, we do not
comment on multiple testing issues in more detail following a suggestion by
the editor of TAS.
 
\item \textsl{pages 12-14:  There are several details about the "photo-carcinogenicity experiments" 
              that are unclear. Access to the data would help. Were there 36 animals (rats? mice?) 
              in each group? Also, at the end of all these tests, did we learn anything more than the 
              figures told us up-front? 
              In addition, if groups A and B have similar survival times and times to first tumor, 
              as the figures show, how do we see these results with the test statistics? }

Details on sample sizes and animals have been added. All data sets analyzed
here are contained in the \texttt{coin} package, i.e., \texttt{data("photocar", package =
"coin")} gives you full control. Moreover, one of the \texttt{coin} package
vignettes contains the complete \textsf{R} sources of our manuscript, i.e.,
all research results reported in this manuscript are fully reproducible.

Comparing group differences by multiple testing procedures is complicated by
the fact that subset pivotality does not hold under those circumstances and
thus one would have to go into the details of multiple testing here. As
noted above, we followed a suggestion by the editor not to do so in order to
keep the focus clear. Parts of this analysis requiring a background in
multiple testing have been removed.

\item \textsl{pages 14-16:  Likewise, there are several details about the "contaminated fish 
              consumption" example that are unclear. What were the sample sizes? What is the 
              "coherence criterion?" Why does the partially ordered comparison make biological 
              sense? What did these tests tell us about the subject-matter example? How do we 
              interpret Figure 5? }

We added the required information to the problem description. Figure 5 can
be used to judge the quality of the normal approximation (compared to the
true `exact' distribution, please see point \ref{asympt} in addition).
 
\item \textsl{pages 16-17:  The discussion provides a fairly accurate summary of this paper. 
              Still, one must have reservations about a highly automated conditional inference 
              system where "the burden of implementing a Monte-Carlo procedure, or even thinking 
              about asymptotics, is waived." Some good diagnostic procedures for this last 
              assumption would be reassuring. }
 
Our formulation regarding the limiting distribution was misleading.
We only claim that one \textit{can} compute such an approximation with the
tools described in the manuscript, however, its quality needs to be
assessed by the data analyst (and we included Figure 5 for this reason).

\item \textsl{This paper has relatively few grammatical errors, and all should be found when 
              revising this paper.}

fixed (hopefully).

\end{enumerate}

\end{document}
  
