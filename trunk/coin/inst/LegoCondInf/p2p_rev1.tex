
\documentclass[11pt]{article}
\usepackage{hyperref}
\usepackage{amstext}
\usepackage{amsfonts}
\usepackage{graphicx}
\usepackage[round]{natbib}

\renewcommand{\baselinestretch}{1.2}

\begin{document}

\title{Point to Point Answers for Manuscript \\
`A Lego System for Conditional Inference' \\
TAS MS05-239, Reviewer \#1}
\author{Anonymous}
\maketitle

\begin{enumerate}
\item \textsl{The ideas raised in this paper are interesting. However, the presentation of the 
              material lacks sufficient explanations for the interested or committed reader. 
              Many of the sentences have compound ideas that beg for further elaboration. 
              This is especially important because the two main sources (Strasser and Weber, 1999; 
              Janssen and Pauls, 2003) are not physically or intellectually accessible to the 
              typical reader (Mathematical Methods of Statistics is a Russian Journal, and I have 
              had to order this background paper from my library). The subject-matter examples are 
              missing important details and the numerical results are not always explained in the 
              original context. It is often not clear what one gain with the methods presented here 
              that would not be obtained with standard methods.}

\item \textsl{The documentation for the new R-procedure \texttt{independence\_test()} is also just not 
              adequate. The paper needs to state exactly what each input and output from the 
              software means. Also, do not assume that the typical reader will be willing to 
              read through S-type or R-type code or output without careful guidance. Most will 
              simply gloss over the code or output.}

\item \textsl{As a side point, is this R-procedure appropriately named, or should it be called 
              something else that reflects the conditional nature of the inference?}

\item \textsl{It should also be noted that 'Lego' is a registered trademark and should not be used 
              as an adjective to describe a procedure. Furthermore, there is nothing about the 
              proposed procedure that makes one think specifically of Lego building blocks, 
              let alone any other building blocks (beyond the most metaphorical sense). The 
              metaphor of the 'spanner' also seems stretched.}

\item \textsl{page 1:  What is meant by "non-standard inference procedures" here? The classical 
              approaches seem pretty standard.}

\item \textsl{page 2:  The introduction needs to be much clearer. Make it clear what is meant by 
              "certain null hypotheses." For example, there are null hypotheses that specify the 
              underlying cell proportions in a contingency table, and there are null hypotheses 
              that specify only independence between rows and columns, but not the proportions. 
              Also, not all conditional tests are permutation tests; there are both conditional 
              exact permutation tests and conditional asymptotic tests. Also, mention some of 
              the pros and cons of permutation tests explicitly here.}

\item \textsl{page 3:  The Wilcoxon-Mann-Whitney and Cochran-Mantel-Haenszel tests are not 
              so much inflexible as specific to particular problems. It is indeed nice that they 
              can be generalized, but this in no way detracts from their value or utility, where 
              appropriate. Conversely, generalized methods lack some of the specificity of 
              simpler methods and become hard to interpret.}

\item \textsl{page 4:  Elaborate on "permutation symmetric way" here, meaning the value of the 
              function does not depend on the order in which the Y-variables appear.}
 
\item \textsl{page 5:  Regarding the last sentence, "Less well known ..." It is not obvious that 
              the conditional distribution can be approximated by its limiting distribution 
              under - all - circumstances, but in any case the accuracy of this approximation 
              is what is of most concern. The implications of this and the theory in the rest of 
              this paragraph should be developed more clearly. }
 
\item \textsl{pages 6-8:  In the "genetic components of alcoholism" example, some essential details are 
              missing from the text, such as the sample size in each group, that there are three allele 
              lengths (short, intermediate, long), and how this relates to the case-control study from 
              which the data were taken, etc. Some of these details can be figured out from the figure 
              and subsequent text, but the logical flow is awkward.  As for the R-code, it would help 
              to include explicitly the formula statement: "formula = elevel ~ alevel". The "trafo" 
              statement is not explained either.  As for the alternative approach, why were these 
              particular scores (2, 7, 11) chosen, other than that they were chosen before? 
              These choices seem to be data driven. 
              Using the Jonkheere-Terpestra test seems to be a better way to get at the ordering of 
              these allele levels. Can this test be implemented in the new procedure? 
              Finally, it is not advisable to compare methods based on p-values, since they measure 
              the statistical significance of effects with respect to particular hypotheses, but 
              ignore other factors, like absolute magnitude, that may also be important. After all, 
              both p-values are pretty small.}

\item \textsl{page 9:  In the "smoking and Alzheimer's disease" example, the paper assumes that the 
              readers will recognize the nesting structure (within gender) in the R-code. Also, 
              why did the "ytrafo" and "trafo" statements disappear? }
 
\item \textsl{page 10:  It would help to state in plain English that the R-code restricts the 
              analysis to a particular subset (males, females).}
 
\item \textsl{page 11:  The paper needs to explain the calculation of the quantiles of the 
              permutation distribution better. Why is the 95% quantile (2.81) so large compared to the 
              usual two-sided normal-score of 1.96 (and is this the correct comparison)? Also, 
              the paper needs to explain how the p-values were adjusted for multiple testing 
              (Bonferroni?).}
 
\item \textsl{pages 12-14:  There are several details about the "photo-carcinogenicity experiments" 
              that are unclear. Access to the data would help. Were there 36 animals (rats? mice?) 
              in each group? Also, at the end of all these tests, did we learn anything more than the 
              figures told us up-front? 
              In addition, if groups A and B have similar survival times and times to first tumor, 
              as the figures show, how do we see these results with the test statistics? }

\item \textsl{pages 14-16:  Likewise, there are several details about the "contaminated fish 
              consumption" example that are unclear. What were the sample sizes? What is the 
              "coherence criterion?" Why does the partially ordered comparison make biological 
              sense? What did these tests tell us about the subject-matter example? How do we 
              interpret Figure 5? }
 
\item \textsl{pages 16-17:  The discussion provides a fairly accurate summary of this paper. 
              Still, one must have reservations about a highly automated conditional inference 
              system where "the burden of implementing a Monte-Carlo procedure, or even thinking 
              about asymptotics, is waived." Some good diagnostic procedures for this last 
              assumption would be reassuring. }
 
\item \textsl{This paper has relatively few grammatical errors, and all should be found when 
              revising this paper.}

\end{enumerate}

\end{document}
  
 
 
 
 
 
 
 
 
 
 
 
 
 
 
 
 

    [ Part 2, Application/PDF  19KB. ]
    [ Unable to print this part. ]

