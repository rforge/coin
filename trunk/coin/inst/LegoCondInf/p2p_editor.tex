
\documentclass[11pt]{article}
\usepackage{hyperref}
\usepackage{amstext}
\usepackage{amsfonts}
\usepackage{graphicx}
\usepackage[round]{natbib}

\renewcommand{\baselinestretch}{1.2}

\begin{document}

\title{Point to Point Answers for Manuscript \\
`A Lego System for Conditional Inference' \\
TAS MS05-239, Editor / Associate Editor}
\author{}
\maketitle

\begin{enumerate}

\item \textsl{Are the assumptions needed for Lego clearly specified enough? 
              The readers can find the background papers, but as mentioned by a reviewer, 
              they are in obscure places.  Therefore, the article must be more self-contained, 
              including specific assumptions.  For example, are the $(Y_i, X_i)$ pairs assumed independent?  
              Later, blocks are discussed -- what is being assumed here, specifically?}

We extended our sketch of the theory following the above suggestions and the
suggestions of referee \#1 at various places, including requirements like
independence of observations and blocks or permutation symmetry of the
influence function. 

\item \textsl{Another concern is that the Westfall-Young method is valid for permutation tests 
              only when subset pivotality holds -- a reasonable assumption for multivariate 
              endpoints but not for pairwise comparisons.  While the Lego system can undoubtedly be 
              extended to general multiple comparisons via closure (which reduces to Westfall-Young 
              for certain types of statistics when subset pivotality holds), that would best be a 
              topic for another paper.  I would therefore suggest dropping MCPs from this paper 
              in the interest of keeping the message more focused.}

Thank you very much for this suggestion. Indeed, explaining the multiple
testing issues in our applications was extremely challenging and we feel
that the manuscript became clearer after omitting the complicated multiple
testing parts. 

\item \textsl{The examples in Section 3 are currently highlighted by the application area 
              such as "Genetic Components of Alcoholism". The authors may consider adding the 
              statistical method illustrated (for example, Kruskal-Wallis test) so that the 
              reader can more easily locate the needed R code.}

Directly adding the name of the
corresponding `classical' test has the disadvantage that it either might
reflect some kind of `cook book' approach (which we would like to avoid
here) or that there is no classical and well known test available (third and
fourth application). We chose to add the inference problem (independent K
samples, contingency tables, multivariate response and independent two
samples) to the titles of the paragraph which should help the interested
reader to identify the type of inference problem dealt with more easily.
 

\end{enumerate}

\end{document}
