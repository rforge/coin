
\documentclass[11pt]{article}
\usepackage{hyperref}
\usepackage{amstext}
\usepackage{amsfonts}
\usepackage{graphicx}
\usepackage[round]{natbib}

\renewcommand{\baselinestretch}{1.2}

\begin{document}

\title{Point to Point Answers for Manuscript \\
`A Lego System for Conditional Inference' \\
TAS MS05-239, Reviewer \#2}
\author{Anonymous}
\maketitle

\begin{enumerate}

\item \textsl{It would be interesting for the authors to list the Monte Carlo equations
              (as mentioned on pages 5 and 6) that correspond to those that they have 
              presented from Strasser and Weber, and to see how the two sets of equations 
              compare to each other.}

We used the term `Monte-Carlo' in two different meanings and thus the last
paragraph of Section~2 became unclear and misleading. For the first part, we
simply meant shuffling of the $Y$-values on the fixed $X$-values (now
termed `resampling') which has nothing in common with quasi-randomized
Monte-Carlo procedures for the approximation of a multivariate normal
distribution. This has been clarified in the paper.

\item \textsl{Page 6 – 5 lines from the bottom – change “every day’s data analysis” to
              “typical data analysis” or “ordinary data analysis”}

changed.

\item \textsl{Page 14 – 10 lines from the bottom – change “third application” to
              “fourth application”}

fixed, thanks!

\item \textsl{The authors present four interesting examples from previously published
              papers or texts. The authors should also include one or two examples 
              based on simulated data using pre-specified
              parameters or contrived data that are fixed in advance to obtain a specific
              result.}

It would indeed be very interesting to study and illustrate the behavior of the
permutation tests that can be performed using \texttt{coin} by means of
simulation experiments in well known (e.g. permutation $t$ test) and new
challenging situations. However, following a suggestion by referee \#1, we
decided to add more details on the underlying theory to the manuscript
which, at least for the interested reader, can serve as a guide to both
theoretical conclusions and self-designed simple simulation experiments. 

However, it is extremely useful to check the results of implementations
against settled knowledge for quality assurance reasons by means of
simulations. We
think that this is beyond the scope of our manuscript and would like to
refer referee \#2 to the \texttt{coin/tests} directory in the source
package, which contains extensive quality assurance procedures comparing our
implementations against well known results.

\item \textsl{The authors should attempt to address whether the theoretical concepts of
              Strasser and Weber can be translated into software routines using 
              SAS (PROC IML) and Stata (the MATA matrix programming language). In other 
              words, can software implementation of flexible and adjustable
              spanners be carried out in SAS and/or Stata? This will be of interest to
              journal readers since there are many more statisticians who regularly use 
              SAS and Stata than who regularly use R.}

In principle, one can implement functions for computing linear statistics
and tools for corresponding tests in any programming language. The design of
the \texttt{coin} implementation heavily depends on concepts like lexical
scoping and functions being first class objects one can compute on. For
example, the function \texttt{independence\_test} does not only compute a
fixed $p$-value but returns a \textit{function} which can be used to
evaluate the complete distribution function.

To our knowledge, such constructs are currently only available in two
statistical languages (\textsf{XLispStat} and \textsf{R}) and thus a direct
`translation' to \textsf{SAS/IML} would be burdensome. However, we would
very much welcome such an implementation and it is our hope that the
publication of this manuscript will be a motivation to transfer our ideas
to other computing environments. We feel that an explicit note on those
issues could be misinterpreted (since some readers might have very strong 
feelings about their preferred analysis system) and so we tend to omit 
such a statement.

\end{enumerate}

\end{document}
