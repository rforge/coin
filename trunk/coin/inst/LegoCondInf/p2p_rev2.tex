
\documentclass[11pt]{article}
\usepackage{hyperref}
\usepackage{amstext}
\usepackage{amsfonts}
\usepackage{graphicx}
\usepackage[round]{natbib}

\renewcommand{\baselinestretch}{1.2}

\begin{document}

\title{Point to Point Answers for Manuscript \\
`A Lego System for Conditional Inference' \\
TAS MS05-239, Reviewer \#2}
\author{Anonymous}
\maketitle

\begin{enumerate}

\item \textsl{It would be interesting for the authors to list the Monte Carlo equations
              (as mentioned on pages 5 and 6) that correspond to those that they have 
              presented from Strasser and Weber, and to see how the two sets of equations 
              compare to each other.}

\item \textsl{Page 6 – 5 lines from the bottom – change “every day’s data analysis” to
              “typical data analysis” or “ordinary data analysis”}

\item \textsl{Page 14 – 10 lines from the bottom – change “third application” to
              “fourth application”}

\item \textsl{The authors present four interesting examples from previously published
              papers or texts. The authors should also include one or two examples 
              based on simulated data using pre-specified
              parameters or contrived data that are fixed in advance to obtain a specific
              result.}

\item \textsl{The authors should attempt to address whether the theoretical concepts of
              Strasser and Weber can be translated into software routines using 
              SAS (PROC IML) and Stata (the MATA matrix programming language). In other 
              words, can software implementation of flexible and adjustable
              spanners be carried out in SAS and/or Stata? This will be of interest to
              journal readers since there are many more statisticians who regularly use 
              SAS and Stata than who regularly use R.}

\end{enumerate}

\end{document}
  
 
 
 
 
 
 
 
 
 
 
 
 
 
 
 
 

    [ Part 2, Application/PDF  19KB. ]
    [ Unable to print this part. ]

